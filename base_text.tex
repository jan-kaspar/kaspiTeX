\input base

%----------------------------------------------------------------------------------------------------

\newif\ifToc
\newif\ifIdx
\newif\ifRef

\newif\ifLocalEqnNumbering

% whether to show labels of equations, figures, tables, etc.
\newif\ifShowLabels

%----- output files ---------------------------------------------------------------------------------

\newwrite\TOC
\newwrite\IDX
\newwrite\REF

\def\MakeRef#1#2{%
	\makecom{#1}{#2}%
	\ifRef
		\immediate\write\REF{\string\makecom{#1}{#2}}%
	\fi
}

%----- counters & dimensions ------------------------------------------------------------------------

% equations, tables, figures...
\newcount\eqn		\eqn		= 0
\newcount\tabn		\tabn		= 0				% per chapter
\newcount\gtabn		\gtabn		= 0				% globally
\newcount\fign		\fign		= 0
\newcount\bookn		\bookn		= 0
\newcount\bibn		\bibn		= 0
\newcount\reffirstn	\reffirstn	= 0
\newcount\nftn		\nftn		= 0				% footnotes counting
\newcount\abbn		\abbn		= 0				% abbreviations

\newcount\iter									% generic iterator

\newdimen\pageamp	\pageamp 	= 0in			% left-right page offset amplitude

%----- page layout ----------------------------------------------------------------------------------

% settings of csplain, fitted to A4 = 210 mm  x  297 mm
% 159.2 mm + 1 in + 1 in = 210 mm
% 239.2 mm + 1 in + 1 in + 7 mm = 297 mm

\vsize=239.2mm
\hsize=159.2mm
\advance\voffset by 3.5mm	% to make upper and bottom margin equally high

\def\forceoddpage{\ifodd\pageno\else{\def\makeheadline{}\vbox to0pt{}\vfil\eject}\fi}

%----- output routine -------------------------------------------------------------------------------

\def\Output{\ifodd\pageno\hoffset=\pageamp\else\hoffset=-\pageamp\fi\plainoutput}
\output={\Output}

%----- global commands ------------------------------------------------------------------------------

\def\BeginText{%
	\ifRef
		\input\jobname.ref
		\immediate\openout\REF\jobname.ref%
	\fi
	\pageno=1
}

\def\EndText{%
	\ifRef
		\immediate\closeout\REF
	\fi
}

%----- captions -------------------------------------------------------------------------------------

\def\CaptionPrefix{}

\def\TableCap{{\bf Table \CaptionPrefix\the\tabn}:}
\def\TableRef{\CaptionPrefix\the\tabn}
\def\FigureCap{{\bf Figure \CaptionPrefix\the\fign}:}
\def\FigureRef{\CaptionPrefix\the\fign}
\def\EqCap{(\CaptionPrefix\the\eqn)}
\def\EqRef{\CaptionPrefix\the\eqn}

\def\Eq#1{\ForAllComSepImp{#1}{\ref}{Eq\hbox{.}~}{Eqs\hbox{.}~}{, }{ and }}
\def\Er#1{\ForAllComSepImp{#1}{\ref}{}{}{, }{ and }}
\def\Fg#1{\ForAllComSepImp{#1}{\fref}{Fig\hbox{.}~}{Figs\hbox{.}~}{, }{ and }}
\def\Tb#1{\ForAllComSepImp{#1}{\tref}{Tab\hbox{.}~}{Tabs\hbox{.}~}{, }{ and }}

%----- tables ---------------------------------------------------------------------------------------


\def\MakeTableCaption{\IfNch[{\MakeTCS}{\MakeTC}}
	
\def\MakeTC#1\endTableCaption#2{%
	\vbox{%
		\CompileTableCaption{#1}{#2}%
	}%
}

\def\MakeTCS[#1]#2\endTableCaption#3{%
	\def\temp{#1}%
	\vbox{%
		\ifx\temp\empty
			\hsize\wd0
		\else
			\hsize\temp
		\fi
		\CompileTableCaption{#2}{#3}%
	}%
}%

\def\CompileTableCaption#1#2{%
	\noindent
	\ifShowLabels
		\vbox to 0pt{\vss\rlap{\labelColor#2\cFg}\hbox{\strut}}%
	\fi
	\strut\TableCap{} % space on purpose
	#1
}

% normal (vertical) table - header cells are above columns
\def\VerticalTableTemplate{%
	\bstrut\bvrule\hfil\tskip $##$\tskip\hfil&&
	\vrule\hfil\tskip $##$\tskip\hfil\cr
}

% horizontal table - header cells are left of rows
\def\HorizontalTableTemplate{%
	\bstrut\bvrule\hfil\tskip $##$\tskip\hfil&
	\bstrut\bvrule\hfil\tskip $##$\tskip\hfil&&
	\vrule\hfil\tskip $##$\tskip\hfil\cr
}

% makes a table reference label
\def\tlab#1{%
	\MakeRef{tab:#1}{\TableRef}
	\pdfdest name {tab:#1} xyz
	\ifx\pdfoutput\undefined
		\special{src:tab:"#1"}%
	\fi
}

% makes a table
% #1 = reference label, #2 = caption, #3 = template, #4 = table data
% #5: if non-empty, the rhs line will be drawn
\def\MakeTable#1#2#3#4#5{%
	\bgroup
		\def\P{#5}
		\ifx\P\empty
			\setbox0=\hbox{\vbox{\offinterlineskip\halign{\span#3#4}}}%
		\else
			\setbox0=\hbox{\vbox{\offinterlineskip\halign{\span#3#4}}\bvrule}%
		\fi
		\global\advance\tabn1
		\global\advance\gtabn1
		\makecom{gtabnum:\the\gtabn}{\the\nchapter.\the\tabn}%
		\makecom{gtabpage:\the\gtabn}{\the\pageno}%
		\makecomNonEx{gtabdesc:\the\gtabn}{#2}%
		\vtop{%
			\tlab{#1}%
			\pdfdest name {gtab:\the\gtabn} xyz
			\hbox{\hss\copy0\hss}%
			\hbox to\wd0{%
				\hss
				%\SmallerFonts
				\MakeTableCaption#2\endTableCaption{#1}\hss
			}%
		}%
		\hss
	\egroup
}

% row-centered layout
\def\mcitm#1{\hbox to0pt{\hss#1\hss}\hfil}                   					% an element of centering
\def\bmcent{\removelastskip\vskip4mm\noindent\line\bgroup\hss}   				% beginning of centering
\def\emcent{\hskip0pt minus1fil \egroup\vskip4mm}								% end of centering

% horizontal table
\def\htab#1#2#3{\bmcent\MakeTable{#1}{#2}{\HorizontalTableTemplate}{#3}{Y}\emcent}

% vertical table
\def\tabAuto#1#2#3{\bmcent\MakeTable{#1}{#2}{\VerticalTableTemplate}{#3}{Y}\emcent}

% table with manual template
\def\tabMan[#1]#2#3#4{\bmcent\MakeTable{#2}{#3}{#1}{#4}{}\emcent}

\def\tab{\IfNch[{\tabMan}{\tabAuto}}

\def\inline#1#2{\omit #1\hfil\ \vbox to0pt{\vss\hbox{#2}\vss}\ \hfil}			% policko nulove vysky, bez pouziti zahlavi

% makes a table reference
\def\tref#1{%
	\pdfstartlink goto name {tab:#1}%
	\linkColor%
	\safecom{tab:#1}%
	\cFg%
	\pdfendlink
}

% vertical joining of cells
\def\mc#1{\hfil\vbox to0pt{\vss\hbox{$#1$}\vss}\hfil}

% printing the list of tables
\def\PrintListOfTables{%
	\iter=0
	\loop\ifnum\iter<\gtabn
		\advance\iter1
		\PrintLOTLine{\the\iter}%
	\repeat
}

\def\PrintLOTLine#1{%
	\vskip2mm
	\indent
	\pdfstartlink goto name {gtab:#1}%
	\em{Table \safecom{gtabnum:#1} on page \safecom{gtabpage:#1}}: \safecom{gtabdesc:#1}%
	\pdfendlink
}

%----- figures --------------------------------------------------------------------------------------

\def\SingleFigTit#1#2{%
	\bgroup
	%\SmallerFonts
	\ifx\bmfigCap\empty	% if bmfigCap is empty make full caption Figure n+1: ...
		\ifShowLabels
			\vbox to 0pt{\vss\rlap{\labelColor#2\cFg}\hbox{\strut}}%
		\fi	
		\FigureCap{} %
		#1%
	\else
		#1
	\fi
	\egroup
}

\def\InsertFiguerTitleNW#1\endFigureTitle#2{%
	\hbox to\wd0{%
		\hss
		\vbox{%
			\leftskip0pt
			\rightskip0pt
			\parfillskip0pt plus 1fil
			\noindent\SingleFigTit{#1}{#2}%
		}%
		\hss
	}%
}

\def\InsertFigureTitle[#1]#2\endFigureTitle#3{%
	\def\temp{#1}%
	\hbox to\wd0{%
		\hss\vbox{%
			\ifx\temp\empty
				\hsize\wd0
			\else
				\hsize\temp
			\fi
			\leftskip0pt
			\rightskip0pt
			\parfillskip0pt plus 1fil
			\noindent\SingleFigTit{#2}{#3}%
		}\hss
	}%
}

\def\fiigSbase#1#2#3#4#5{
	\SmallerFonts%
	\setbox1=\hbox{%
		\vbox to0pt{\vss\RotCCW{#3}\vss}%
		\vbox to0pt{\vss\copy0\vss}%
		\vbox to0pt{\vss\RotCCW{#5}\vss}%
		}%
	\setbox0=\vbox to\ht0{\vss\copy1\vss}%
	\vtop{%
		\hbox to\wd0{\hss #4\hss}%
		\copy0%
		\def\temp{#2}%
		\ifx\temp\empty
		\else	% insert title only if not emtpy, empty title produces a one-line vspace
			\IfNch[{\InsertFigureTitle}{\InsertFiguerTitleNW}#2\endFigureTitle{#1}
		\fi
		}%
	%\ifx\bmfigCap\empty
	%	\flab{#1}% make label
	%\fi
}


% special (*), no width
\def\fiigSNW*#1#2#3#4#5#6{%
	\hfil
	\bgroup%
		\setbox0=\hbox{\IncludeGraphics{#1}}%
		\fiigSbase{#2}{#3}{#4}{#5}{#6}%	
	\egroup
}
	
% special (*), width
\def\fiigSW*[#1]#2#3#4#5#6#7{%
	\hfil
	\bgroup
		\setbox0=\hbox{\IncludeSizedGraphics{#1}{#2}}%
		\fiigSbase{#3}{#4}{#5}{#6}{#7}%
	\egroup
}
	
% not special (*), no width
\def\fiigNW#1#2#3{%
	\hfil
	\setbox0=\hbox{\IncludeGraphics{#1}}%
	\vtop{%
		\global\advance\fign1%
		\flab{#2}%
		\copy0%
		\IfNch[{\InsertFigureTitle}{\InsertFiguerTitleNW}#3\endFigureTitle{#2}
	}%
}

% not special (*), width
\def\fiigW[#1]#2#3#4{%
	\hfil
	\setbox0=\hbox{\IncludeSizedGraphics{#1}{#2}}%
	\vtop{%
		\global\advance\fign1%
		\flab{#3}%
		\copy0%
		\IfNch[{\InsertFigureTitle}{\InsertFiguerTitleNW}#4\endFigureTitle{#3}
	}%
}

	
\def\fiigS*{\IfNch[{\fiigSW}{\fiigSNW}}	
\def\fiigNS{\IfNch[{\fiigW}{\fiigNW}}	
\def\fiig{\IfNch*{\fiigSW}{\fiigNS}}	

\def\figSNW#1#2#3#4#5#6{\bmfig\fiigSNW*{#1}{#2}{#3}{#4}{#5}{#6}\emfig}
\def\figSW[#1]#2#3#4#5#6#7{\bmfig\fiigSW*[#1]{#2}{#3}{#4}{#5}{#6}{#7}\emfig}
\def\figNW#1#2#3{\bmfig\fiigNW{#1}{#2}{#3}\emfig}
\def\figW[#1]#2#3#4{\bmfig\fiigW[#1]{#2}{#3}{#4}\emfig}

\def\figS*{\IfNch[{\figSW}{\figSNW}}	
\def\figNS{\IfNch[{\figW}{\figNW}}	
\def\fig{\IfNch*{\figS}{\figNS}}	


% labeling
\def\flab#1{%
	\def\temp{#1}%
	\ifx\temp\empty
	\else
		\ifx\pdfoutput\undefined
			\special{src:fig "#1"}
		\else
			\pdfdest name {fig:#1} xyz
		\fi
		\MakeRef{fig:#1}{\FigureRef}%
	\fi
}

% references
\def\fref#1{%
	\pdfstartlink goto name {fig:#1}%
	\linkColor%
	\safecom{fig:#1}%
	\cFg%
	\pdfendlink
}

% several figures in a row
\def\bmfigCap{}		% caption of bmfig
\def\bmfig{\IfNch[{\bmfigS}{\bmfigNS}}
\def\bmfigNS{\bmfigS[]}
\def\bmfigS[#1]{%
	\bgroup\let\fig\fiig%
	\removelastskip\vskip4mm%
	\def\bmfigCap{#1}% inside group, important!, outside the bmfigCap must be empty to make captions for single figures
	\vbox\bgroup%
		\leftskip0pt plus1fil
		\rightskip0pt plus1fil
		\parfillskip0pt
		\noindent\hskip0pt plus-1fil
}
				
\def\emfig{%
	\egroup
	\leftskip0pt
	\ifx\bmfigCap\empty\else%
		\global\advance\fign1%
		\centerline{\SmallerFonts\vbox{\advance\hsize-1cm \noindent \FigureCap{} \bmfigCap}}%
	\fi%
	\egroup% group with SmallerFonts etc.
	\vskip4mm
}

%----- equations ------------------------------------------------------------------------------------

\def\eq#1{%
	\ifLocalEqnNumbering
		\advance\eqn1
	\else
		\global\advance\eqn1
	\fi
	$$#1\eqno{\hbox{\EqCap}}$$%
}

\def\eqref#1#2{%
	\ifx\pdfoutput\undefined\special{src:eq:"#2"}\fi
	\ifLocalEqnNumbering
		\advance\eqn1
	\else
		\global\advance\eqn1
	\fi
	\MakeRef{eq:#2}{\EqRef}%
	\pdfdest name {eq:#2} xyz
	\ifShowLabels
		$$#1\eqno{\hbox{\EqCap}\rlap{\ \labelColor#2\cFg}}$$%
	\else
		$$#1\eqno{\hbox{\EqCap}}$$%
	\fi
	%
}

\def\eqnoref{\IfNch[{\eqnorefMan}{\eqnorefStd}}
\def\eqnorefStd#1{%
	\global\advance\eqn1
	\MakeRef{eq:#1}{\EqRef}%
	\eqno{\hbox{\EqCap}}%
}
\def\eqnorefMan[#1]#2{%
	\MakeRef{eq:#2}{#1}%
	\eqno{\hbox{\rm #1}}%
}

\def\ref#1{%
	\pdfstartlink goto name {eq:#1}%
	\linkColor%
	\safecom{eq:#1}%
	\cFg%
	\pdfendlink
}

%----- abbreviation management ----------------------------------------------------------------------

\def\MakeAbbRef#1{%
	\ea\ifx\csname abbref:#1\endcsname\relax%
		\global\advance\abbn1%
		\makecom{abbref:#1}{\the\abbn}%
		\makecom{abbrefnum:\the\abbn}{#1}%
	\fi
}

\def\abb#1{%
	\leavevmode
	\MakeAbbRef{#1}%
	%\pdfstartlink goto name {abb:\csname abbref:#1\endcsname}%
	\pdfstartlink goto name {abb:#1}%
	\linkColor#1\cFg%
	\pdfendlink
}

\def\PrintAbbreviations{%
	\iter=0
	\loop\ifnum\iter<\abbn
		\advance\iter1
		\PrintAbbreviation{\the\iter}
	\repeat
}

\def\PrintAbbreviation#1{%
	\vskip2mm
	\pdfdest name {abb:#1} xyz
	\noindent
	\hbox to25mm{%
		\bf
		\safecom{abbrefnum:#1}%
		\hfil
	}%
	\vtop{%
		\advance\hsize-25mm
		\noindent
		\safecom{abbdef:#1}%
	}%
}

\def\DefAbb#1#2#3{%
	\MakeAbbRef{#1}%
	\def\temp{#3}%
	\ifx\temp\empty
		\edef\temp{{\it#2}}%	
	\else
		\edef\temp{{\it#2}. #3}%	
	\fi
	\makecom{abbdef:\csname abbref:#1\endcsname}{\temp}%
}

\def\ListAbb#1#2#3{%
	\vskip2mm
	\pdfdest name {abb:#1} xyz
	\noindent
	\hbox to16mm{%
		\bf
		#1
		\hfil
	}%
	\def\short{#2}%
	\def\long{#3}%
	\vtop{%
		\advance\hsize-16mm
		\noindent
		\ifx\short\empty
			#3%
		\else
			\ifx\long\empty
				{\it#2}%
			\else
				{\it#2}. #3%
			\fi
		\fi
	}%
}

%----- footnotes ------------------------------------------------------------------------------------

\let\oldfootnote\footnote
\def\footnote#1{%
	\bgroup
		\global\advance\nftn1
		\SmallerFonts
		\oldfootnote{$^{\,\the\nftn)}$}{#1}%
	\egroup
}
